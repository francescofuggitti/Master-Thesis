\chapter{Janus}
In this chapter, we will illustrate how our tool \LTLfToDFA presented in Chapter \ref{ch:ltlf2dfa} can be efficiently employed in the field of Business Process Management, with particular attention to Process Mining. First of all, we will formally describe the theoretical framework of declarative process mining, introducing a new theorem that generalizes the concept of separated formulas only for \declare constraints. Then, in this context, we will thoroughly describe the implementation of the Janus algorithm \citep{cecconi2018interestingness} for computing the interestingness degree of traces in real event logs. Finally, we will provide such a computation for a real log. 
\section{Preliminaries}
In this section, we will present the theoretical framework of Business Process Management focusing our attention to declarative process mining. We will extend what described in Chapter \ref{ch:theory} providing all additional concepts, definitions and theorems necessary to clearly understand the context.

Business Process Management (BPM) deals with discovering, modeling, analyzing and managing business processes in order to measure their productivity and to improve their performance. These tasks are carried out thanks to logging facilities that, nowadays, all BPM systems have. The extraction and the validation of temporal constraints from event logs (i.e. multi-sets of finite traces) are techniques used in declarative process mining \citep{montali2010declarative}.
\subsection{Event Logs}
Declarative Process Mining
\subsection{\declare}
\subsection{Separation Theorem}
\section{Janus}
\subsection{Idea/Approach}
\subsection{\LTLfToDFA in Janus}
\subsection{Algorithm}
\section{Implementation}
\subsection{Package Structure}
\subsection{Classes}
\section{Summary}